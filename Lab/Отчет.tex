\documentclass[11pt]{article}
\usepackage{amsmath,amssymb,amsthm}
\usepackage{algorithm}
\usepackage[noend]{algpseudocode} 

%---enable russian----

\usepackage[utf8]{inputenc}
\usepackage[russian]{babel}

% PROBABILITY SYMBOLS
\newcommand*\PROB\Pr 
\DeclareMathOperator*{\EXPECT}{\mathbb{E}}


% Sets, Rngs, ets 
\newcommand{\N}{{{\mathbb N}}}
\newcommand{\Z}{{{\mathbb Z}}}
\newcommand{\R}{{{\mathbb R}}}
\newcommand{\Zp}{\ints_p} % Integers modulo p
\newcommand{\Zq}{\ints_q} % Integers modulo q
\newcommand{\Zn}{\ints_N} % Integers modulo N

% Landau 
\newcommand{\bigO}{\mathcal{O}}
\newcommand*{\OLandau}{\bigO}
\newcommand*{\WLandau}{\Omega}
\newcommand*{\xOLandau}{\widetilde{\OLandau}}
\newcommand*{\xWLandau}{\widetilde{\WLandau}}
\newcommand*{\TLandau}{\Theta}
\newcommand*{\xTLandau}{\widetilde{\TLandau}}
\newcommand{\smallo}{o} %technically, an omicron
\newcommand{\softO}{\widetilde{\bigO}}
\newcommand{\wLandau}{\omega}
\newcommand{\negl}{\mathrm{negl}} 

% Misc
\newcommand{\eps}{\varepsilon}
\newcommand{\inprod}[1]{\left\langle #1 \right\rangle}

 
\newcommand{\handout}[5]{
  \noindent
  \begin{center}
  \framebox{
    \vbox{
      \hbox to 5.78in { {\bf Научно-исследовательская практика} \hfill #2 }
      \vspace{4mm}
      \hbox to 5.78in { {\Large \hfill #5  \hfill} }
      \vspace{2mm}
      \hbox to 5.78in { {\em #3 \hfill #4} }
    }
  }
  \end{center}
  \vspace*{4mm}
}

\newcommand{\lecture}[4]{\handout{#1}{#2}{#3}{#4}{Тест на простоту Миллера-Рабина #1}}

\newtheorem{theorem}{Теорема}
\newtheorem{lemma}{Лемма}
\newtheorem{definition}{Определение}
\newtheorem{corollary}{Следствие}
\newtheorem{fact}{Факт}

% 1-inch margins
\topmargin 0pt
\advance \topmargin by -\headheight
\advance \topmargin by -\headsep
\textheight 8.9in
\oddsidemargin 0pt
\evensidemargin \oddsidemargin
\marginparwidth 0.5in
\textwidth 6.5in

\parindent 0in
\parskip 1.5ex

\begin{document}

\lecture{}{Лето 2020}{}{Каминский Владислав}

\section{Описание}
\textbf{Тест Миллера — Рабина} — вероятностный полиномиальный тест простоты. Он позволяет эффективно определить, является ли данное число составным. Однако, с его помощью нельзя строго доказать простоту числа. Тем не менее тест Миллера — Рабина часто используется в криптографии для получения больших случайных простых чисел.
В этом тесте $n-1$ представляется как произведение нечетного числа \textit{m} и степени числа 2. \\Идея теста на простоту числа на основе Ферма заключена в следующем:
   $a^{n-1} = a ^{m2^s} = a^{2m}$.
Другими словами, вместо того чтобы вычислять $a^{n-1}$ в один шаг, мы можем сделать это в \\ \textit{s} шагов. Какое преимущество в таком применении? Преимущество заключается именно в том, что испытание квадратным корнем может быть выполнено на каждом шаге. Если квадратный корень показывает сомнительные результаты, мы останавливаемся и объявляем \textit{n} составным номером. На каждом шаге мы обеспечиваем, что тест Ферма и испытание квадратным корнем удовлетворено на всех парах смежных шагов, если оно удовлетворительно (если результат равен 1).



\section{Алгоритм}

\begin{algorithm}[ph]
	\caption{Тест на простоту Миллера-Рабина}
	\label{alg:AlgName}
	\textbf{Ввод:} n - тестируемое натуральное число; k - количество раундов  проверки  \\
	\textbf{Вывод:} составное; вероятно простое
	
	\begin{algorithmic}[1]
		
		\State найти \textbf{s} и \textbf{m} такие, что $ n - 1 = 2^sm $, где m - нечетно
		\For{$j \gets 0$ to $k - 1$}                    
			\State $a  \gets random(2, n - 1)$
			\State {$x \gets a^{m}$ \textit{mod n}}
			\State \textbf{if} $x = \pm 1$ \textit{mod n} \textbf{then} \Return{вероятно простое} 
			\For{$i \gets 1$ to $s - 1$}   
				\State $x \gets x^2$  \textit{mod n}
				\State \textbf{if} $x = 1$ \textit{mod n} \textbf{then} \Return{составное}
				\State \textbf{else if} $x = - 1$ \textit{mod n} \textbf{then} \Return{вероятно простое}  
			\EndFor
			\State \Return{составное}
		\EndFor
	\end{algorithmic}

\end{algorithm}
\clearpage

\section{Анализ}

\subsection{Описание}
Алгоритм был реализован на \textit{Sage} — системе компьютерной алгебры, покрывающей много областей математики, включая алгебру, комбинаторику, вычислительную математику и матанализ. Вся вычислительная нагрузка приходилась на облачный ресурс Collaborative Calculation and Data Science (Cocalc).

\subsection{Сравнение результатов}

Для проверки корректности реализации алгоритма была использована встроенная в \textit{Sage} функция \textit{is\_prime(n)}. Были зафиксированы результаты работы и среднее время их исполнения.

\begin{center}

\begin{tabular}{|c|c|c|c|c|c|}
\hline
\multicolumn{2}{|c|}{Параметры} & \multicolumn{2}{c|}{Собственная реализация} & \multicolumn{2}{c|}{Sage} \\ \hline
      n     &      k     &    Результат       &     Время, с      &      Результат     &    Время, с       \\ \hline
      $2^{89} - 1$     &     500      &     true      &     0.00      &     true      &    0.00       \\ \hline
       $123456789^{1234} - 987654321$    &     500      &     false     &     5.66      &     false     &     0.01      \\ \hline
       $21021078162...89342215 \sim 10^{150}$    &      500     &     true      &   9.84        &     true      &        0.74    \\ \hline
       $10^{190} + 3001003$    &     500      &     false     &     52.32      &     false     &     0.01      \\ \hline
\end{tabular}
\end{center}


\end{document}